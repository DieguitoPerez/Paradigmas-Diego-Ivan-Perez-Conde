\documentclass{article}
\usepackage[utf8]{inputenc}

\title{Reporte Técnico}
\author{Diego Iván Perez Conde}
\date{March 2021}

\begin{document}

\maketitle

\section{Introduction}

\text{El trabajo que llevare a cabo es una red neuronal que estara echa sobre matlab pero antes de llevar a cabo este proceso hay que empezar con varias definiciones acerca de mi tema, definiremos que es una red neuronal artificial y su objetivo.
Las redes neuronales artificiales (tambien conocidas como sistemas conexionistas) son un modelo computacional el que fue evolucionando a partir de diversas aportaciones cientıficas que estan registradas en la historia. 
Consiste en un conjunto de unidades, llamadas neuronas artificiales, conectadas entre si para transmitirse señales. La información de entrada atraviesa la red neuronal (donde se somete a diversas operaciones) produciendo unos valores de salida.

El objetivo de la red neuronal es resolver los problemas de la misma manera que el cerebro humano, aunque las redes neuronales son más abstractas. Las redes neuronales actuales suelen contener desde unos miles a unos pocos millones de unidades neuronales.
 }
 
\textbf{¿Cómo funcionan las redes neuronales?}

\text{Una red neuronal combina diversas capas de procesamiento y utiliza elementos simples que operan en paralelo, y están inspiradas en los sistemas nerviosos biológicos. Consta de una capa de entrada, una o varias capas ocultas y una capa de salida. Las capas están interconectadas mediante nodos, o neuronas; cada capa utiliza la salida de la capa anterior como entrada.}


\section{Desarrollo}

\text{Para esto el desarrollo de mi tema sera llevado a cabo sobre matlab y sera llevar a cabo figuras geometricas estas que implementare seran 12 figuras para el desarrollo del tema. }

\section{Conclusión}




\end{document}
